\documentclass[10pt,a4paper]{article}
\usepackage[utf8]{inputenc}
\usepackage{amsmath}
\usepackage{amsfonts}
\usepackage{amssymb}
\usepackage{siunitx}
\usepackage{tikz}
\usepackage[margin=1mm]{geometry}
\title{A Method of Experimentally Getting a Value for the Drag Coefficient}
\author{Generalov Danya}
\usetikzlibrary{calc}
\begin{document}
    The drag coefficient is a value inherent in a body moving through a fluid determining the force that counteracts its movement.
    The drag force is defined as:\\


    \begin{equation}
        \label{1}
        F = C_x \frac{\rho v^2}{2}S_x
    \end{equation}

    Where:
    \begin{itemize}
        \item $F$ is the drag force;
        \item $C_x$ is the drag coefficient - what we are looking for;
        \item $\rho$ is the air density;
        \item $S_x$ is the frontal area of the object;
        \item $v$ is the object velocity.
    \end{itemize}

    To calculate the drag coefficient, we need to have the object velocity constant.
    Newton's first law states that an object's velocity is constant if no force is applied to it.
    Therefore, there must be a force counteracting the drag force.\\

    A force that is always present around us is the gravitational pull of the Earth.
    The formula for the force of gravity near the surface of the Earth is:
    \begin{equation}
        \label{2}
        F = mg
    \end{equation}

    Where:
    \begin{itemize}
        \item $F$ is the gravitational force;
        \item $m$ is the mass of the object;
        \item $g$ is the gravitational acceleration, equal to $9.80665 \frac{m}{s^2}$
    \end{itemize}

    Since our object is falling at a constant velocity, combining~\ref{1} and~\ref{2} gives us this formula:\\

    \[C_x \frac{\rho v^2}{2}S_x = mg\]

    \ldots from which follows:
    \begin{equation}
        \label{3}
        C_x = \frac{S_x \rho v^2 mg}{2}
    \end{equation}

    To experimentally find the drag coefficient, therefore, we need an experimental setup where the variables are controlled or monitored.
    For example, if we take an umbrella, which has an aerodynamic shape, open it, then throw it from a tall building whilst pointing a high-resolution, high-framerate camera at the building, then we would be able to measure the fall velocity.
    The reason we would take an umbrella is that its aerodynamic shape would help prevent tumbling, which will change the frontal velocity considerably.
    The mass of the umbrella we can measure by using a scale, and the surface area can be calculated using some geometric trick that I can't be bothered to look up.\\
    \frame{
    \begin{tikzpicture}
        \draw (0,0) -- (0,10) -- (5,10) -- (5,0) -- cycle;
        \draw (4.7,12) circle [radius=0.3];
        \draw (4.7,11.7) -- (4.7,11.0) -- (4.5,10) --(4.7,11.0) -- (4.9,10);
        \draw (4.7,11.6) -- (5.6,11.58);
        \draw (4.7,11.53) -- (5.6,11.52);

        \draw[thick] (5.6,11.58) -- (5.6,10.5);

        \def\centerarc[#1](#2)(#3:#4:#5)% Syntax: [draw options] (center) (initial angle:final angle:radius)
        { \draw[#1] ($(#2)+({#5*cos(#3)},{#5*sin(#3)})$) arc (#3:#4:#5); }
        \centerarc[thick](5.6,11)(-180:0:0.5)
        \draw (4.5,8.5)--(4.5,9.5)--(3.5,9.5)--(3.5,8.5)--cycle;
        \draw (4.5,7)--(4.5,8)--(3.5,8)--(3.5,7)--cycle;

        \draw (4.5,5.5)--(4.5,6.5)--(3.5,6.5)--(3.5,5.5)--cycle;
        \draw (4.5,4)--(4.5,5)--(3.5,5)--(3.5,4)--cycle;

        \draw (4.5,3.5)--(4.5,2.5)--(3.5,2.5)--(3.5,3.5)--cycle;
        \draw (4.5,2)--(4.5,1)--(3.5,1)--(3.5,2)--cycle;


        %---
        \draw (2.5,8.5)--(2.5,9.5)--(3.5,9.5)--(3.5,8.5)--cycle;
        \draw (2.5,7)--(2.5,8)--(3.5,8)--(3.5,7)--cycle;

        \draw (2.5,5.5)--(2.5,6.5)--(3.5,6.5)--(3.5,5.5)--cycle;
        \draw (2.5,4)--(2.5,5)--(3.5,5)--(3.5,4)--cycle;

        \draw (2.5,3.5)--(2.5,2.5)--(3.5,2.5)--(3.5,3.5)--cycle;
        \draw (2.5,2)--(2.5,1)--(3.5,1)--(3.5,2)--cycle;

        %---
        \draw (0.5,8.5)--(0.5,9.5)--(1.5,9.5)--(1.5,8.5)--cycle;
        \draw (0.5,7)--(0.5,8)--(1.5,8)--(1.5,7)--cycle;

        \draw (0.5,5.5)--(0.5,6.5)--(1.5,6.5)--(1.5,5.5)--cycle;
        \draw (0.5,4)--(0.5,5)--(1.5,5)--(1.5,4)--cycle;

        \draw (0.5,3.5)--(0.5,2.5)--(1.5,2.5)--(1.5,3.5)--cycle;
        \draw (0.5,2)--(0.5,1)--(1.5,1)--(1.5,2)--cycle;



        \draw (15,1) -- (15,1.2) -- (15.4,1.2) -- (15.4,1)--cycle;
        \draw (15.2,1) -- (15,0);
        \draw (15.2,1) -- (15.4,0);
        \draw (14.8,1.3)--(15,1.3)--(15,0.9)--(14.8,0.9)--cycle;
        \draw[dashed] (14.8,1.2) -- (5,10);
        \draw[dashed] (14.8,1) -- (5,0);

        \draw (2,12) circle [radius=0.3];
        \draw (2,11.7) -- (2,11.0) -- (1.8,10) --(2,11.0) -- (2.2,10);
        \draw (2,11.6) -- (2.5,11.4)--(2.26,11.83);
        \draw (2,11.53) -- (1.9,11.1)--(2,11.0);
        \node at (2,13) {What are you doing?};
        \draw (1.85,12.3)--(1,12.9);
        \node at (5.9,12.7) {\textbf{\textit{FOR SCIENCE!}}};
        \draw(4.85,12.3) -- (5.9,12.5);
    \end{tikzpicture}} \\


    At first, the drag force will not be sufficient to compensate gravity, and so the umbrella will accelerate downward\ldots\\
    \frame{\begin{tikzpicture}
               \def\centerarc[#1](#2)(#3:#4:#5)% Syntax: [draw options] (center) (initial angle:final angle:radius)
               { \draw[#1] ($(#2)+({#5*cos(#3)},{#5*sin(#3)})$) arc (#3:#4:#5); }
               \draw (0,0) -- (0,10) -- (5,10) -- (5,0) -- cycle;
               \draw (4.7,12) circle [radius=0.3];
               \draw (4.7,11.7) -- (4.7,11.0) -- (4.5,10) --(4.7,11.0) -- (4.9,10);
               \draw (4.7,11.6) -- (5.1,11.3) -- (4.7,11);
               \draw (4.7,11.6) -- (4.3,11.34)--(4.7,11);

               \draw[thick] (5.6,9.58) -- (5.6,8.5);

               \centerarc[thick](5.6,9)(-180:0:0.5)
               \draw (4.5,8.5)--(4.5,9.5)--(3.5,9.5)--(3.5,8.5)--cycle;
               \draw (4.5,7)--(4.5,8)--(3.5,8)--(3.5,7)--cycle;

               \draw (4.5,5.5)--(4.5,6.5)--(3.5,6.5)--(3.5,5.5)--cycle;
               \draw (4.5,4)--(4.5,5)--(3.5,5)--(3.5,4)--cycle;

               \draw (4.5,3.5)--(4.5,2.5)--(3.5,2.5)--(3.5,3.5)--cycle;
               \draw (4.5,2)--(4.5,1)--(3.5,1)--(3.5,2)--cycle;


               %---
               \draw (2.5,8.5)--(2.5,9.5)--(3.5,9.5)--(3.5,8.5)--cycle;
               \draw (2.5,7)--(2.5,8)--(3.5,8)--(3.5,7)--cycle;

               \draw (2.5,5.5)--(2.5,6.5)--(3.5,6.5)--(3.5,5.5)--cycle;
               \draw (2.5,4)--(2.5,5)--(3.5,5)--(3.5,4)--cycle;

               \draw (2.5,3.5)--(2.5,2.5)--(3.5,2.5)--(3.5,3.5)--cycle;
               \draw (2.5,2)--(2.5,1)--(3.5,1)--(3.5,2)--cycle;

               %---
               \draw (0.5,8.5)--(0.5,9.5)--(1.5,9.5)--(1.5,8.5)--cycle;
               \draw (0.5,7)--(0.5,8)--(1.5,8)--(1.5,7)--cycle;

               \draw (0.5,5.5)--(0.5,6.5)--(1.5,6.5)--(1.5,5.5)--cycle;
               \draw (0.5,4)--(0.5,5)--(1.5,5)--(1.5,4)--cycle;

               \draw (0.5,3.5)--(0.5,2.5)--(1.5,2.5)--(1.5,3.5)--cycle;
               \draw (0.5,2)--(0.5,1)--(1.5,1)--(1.5,2)--cycle;



               \draw (15,1) -- (15,1.2) -- (15.4,1.2) -- (15.4,1)--cycle;
               \draw (15.2,1) -- (15,0);
               \draw (15.2,1) -- (15.4,0);
               \draw (14.8,1.3)--(15,1.3)--(15,0.9)--(14.8,0.9)--cycle;
               \draw[dashed] (14.8,1.2) -- (5,10);
               \draw[dashed] (14.8,1) -- (5,0);

               \draw (2,12) circle [radius=0.3];
               \draw (2,11.7) -- (2,11.0) -- (1.8,10) --(2,11.0) -- (2.2,10);
               \draw (2,11.6) -- (2.7,11.5);
               \draw (2,11.53) -- (2.72,11.4);

               \node at (2,13.5) {That was our last umbrella!};
               \draw (1.85,12.3)--(1,13.2);
               \node at (5.9,12.7) {\textit{TOTALLY} worth it.};
               \draw(4.85,12.3) -- (5.9,12.5);

               \draw[->,red,very thick] (5.6,8) -- (5.6,6);
               \node at (6,7) {$F_g$};

               \draw[->,blue,very thick] (5.6,10) -- (5.6,11);
               \node at (6,10.5) {$F_x$};
    \end{tikzpicture}}\\
    \ldots but at some point, the forces will cancel each other out, and the velocity will be constant.\\
    \frame{\begin{tikzpicture}
               \draw (0,0) -- (0,10) -- (5,10) -- (5,0) -- cycle;
               \draw (4.7,12) circle [radius=0.3];
               \draw (4.7,11.7) -- (4.7,11.0) -- (4.5,10) --(4.7,11.0) -- (4.9,10);
               \draw (4.7,11.6) -- (5.1,11.3) -- (4.7,11);
               \draw (4.7,11.6) -- (4.3,11.34)--(4.45,11.85);

               \draw[thick] (5.6,7.58) -- (5.6,6.5);

               \def\centerarc[#1](#2)(#3:#4:#5)% Syntax: [draw options] (center) (initial angle:final angle:radius)
               { \draw[#1] ($(#2)+({#5*cos(#3)},{#5*sin(#3)})$) arc (#3:#4:#5); }
               \centerarc[thick](5.6,7)(-180:0:0.5)
               \draw (4.5,8.5)--(4.5,9.5)--(3.5,9.5)--(3.5,8.5)--cycle;
               \draw (4.5,7)--(4.5,8)--(3.5,8)--(3.5,7)--cycle;

               \draw (4.5,5.5)--(4.5,6.5)--(3.5,6.5)--(3.5,5.5)--cycle;
               \draw (4.5,4)--(4.5,5)--(3.5,5)--(3.5,4)--cycle;

               \draw (4.5,3.5)--(4.5,2.5)--(3.5,2.5)--(3.5,3.5)--cycle;
               \draw (4.5,2)--(4.5,1)--(3.5,1)--(3.5,2)--cycle;


               %---
               \draw (2.5,8.5)--(2.5,9.5)--(3.5,9.5)--(3.5,8.5)--cycle;
               \draw (2.5,7)--(2.5,8)--(3.5,8)--(3.5,7)--cycle;

               \draw (2.5,5.5)--(2.5,6.5)--(3.5,6.5)--(3.5,5.5)--cycle;
               \draw (2.5,4)--(2.5,5)--(3.5,5)--(3.5,4)--cycle;

               \draw (2.5,3.5)--(2.5,2.5)--(3.5,2.5)--(3.5,3.5)--cycle;
               \draw (2.5,2)--(2.5,1)--(3.5,1)--(3.5,2)--cycle;

               %---
               \draw (0.5,8.5)--(0.5,9.5)--(1.5,9.5)--(1.5,8.5)--cycle;
               \draw (0.5,7)--(0.5,8)--(1.5,8)--(1.5,7)--cycle;

               \draw (0.5,5.5)--(0.5,6.5)--(1.5,6.5)--(1.5,5.5)--cycle;
               \draw (0.5,4)--(0.5,5)--(1.5,5)--(1.5,4)--cycle;

               \draw (0.5,3.5)--(0.5,2.5)--(1.5,2.5)--(1.5,3.5)--cycle;
               \draw (0.5,2)--(0.5,1)--(1.5,1)--(1.5,2)--cycle;



               \draw (15,1) -- (15,1.2) -- (15.4,1.2) -- (15.4,1)--cycle;
               \draw (15.2,1) -- (15,0);
               \draw (15.2,1) -- (15.4,0);
               \draw (14.8,1.3)--(15,1.3)--(15,0.9)--(14.8,0.9)--cycle;
               \draw[dashed] (14.8,1.2) -- (5,10);
               \draw[dashed] (14.8,1) -- (5,0);

               \draw (2,12) circle [radius=0.3];
               \draw (2,11.7) -- (2,11.0) -- (1.8,10) --(2,11.0) -- (2.2,10);
               \draw (2,11.6) -- (2.4,11);
               \draw (2,11.53) -- (1.8,11.35)--(2,11.0);

               \node at (2,13.5) {Why are you doing this, again?};
               \draw (1.85,12.3)--(1,13.2);
               \node at (5.9,12.7) {Well\ldots};
               \draw(4.85,12.3) -- (5.9,12.5);

               \draw[->,red,very thick] (5.6,6) -- (5.6,4);
               \node at (6,5) {$F_g$};

               \draw[->,blue,very thick] (5.6,8) -- (5.6,10);
               \node at (6,8.5) {$F_x$};
    \end{tikzpicture}}\\
    At that point, we will measure the distance it has traveled in a period of time, as approximated by the distance between windows, and that will be its velocity.\\
    \frame{\begin{tikzpicture}
               \def\centerarc[#1](#2)(#3:#4:#5)% Syntax: [draw options] (center) (initial angle:final angle:radius)
               { \draw[#1] ($(#2)+({#5*cos(#3)},{#5*sin(#3)})$) arc (#3:#4:#5); }
               \draw (0,0) -- (0,10) -- (5,10) -- (5,0) -- cycle;
               \draw (4.7,12) circle [radius=0.3];
               \draw (4.7,11.7) -- (4.7,11.0) -- (4.5,10) --(4.7,11.0) -- (4.9,10);
               \draw (4.7,11.6) -- (5.1,11.3) -- (5.4,11.6);
               \draw (4.7,11.6) -- (4.3,11.3)--(4,11.6);


               \draw (4.5,8.5)--(4.5,9.5)--(3.5,9.5)--(3.5,8.5)--cycle;
               \draw (4.5,7)--(4.5,8)--(3.5,8)--(3.5,7)--cycle;

               \draw (4.5,5.5)--(4.5,6.5)--(3.5,6.5)--(3.5,5.5)--cycle;
               \draw (4.5,4)--(4.5,5)--(3.5,5)--(3.5,4)--cycle;

               \draw (4.5,3.5)--(4.5,2.5)--(3.5,2.5)--(3.5,3.5)--cycle;
               \draw (4.5,2)--(4.5,1)--(3.5,1)--(3.5,2)--cycle;


               %---
               \draw (2.5,8.5)--(2.5,9.5)--(3.5,9.5)--(3.5,8.5)--cycle;
               \draw (2.5,7)--(2.5,8)--(3.5,8)--(3.5,7)--cycle;

               \draw (2.5,5.5)--(2.5,6.5)--(3.5,6.5)--(3.5,5.5)--cycle;
               \draw (2.5,4)--(2.5,5)--(3.5,5)--(3.5,4)--cycle;

               \draw (2.5,3.5)--(2.5,2.5)--(3.5,2.5)--(3.5,3.5)--cycle;
               \draw (2.5,2)--(2.5,1)--(3.5,1)--(3.5,2)--cycle;

               %---
               \draw (0.5,8.5)--(0.5,9.5)--(1.5,9.5)--(1.5,8.5)--cycle;
               \draw (0.5,7)--(0.5,8)--(1.5,8)--(1.5,7)--cycle;

               \draw (0.5,5.5)--(0.5,6.5)--(1.5,6.5)--(1.5,5.5)--cycle;
               \draw (0.5,4)--(0.5,5)--(1.5,5)--(1.5,4)--cycle;

               \draw (0.5,3.5)--(0.5,2.5)--(1.5,2.5)--(1.5,3.5)--cycle;
               \draw (0.5,2)--(0.5,1)--(1.5,1)--(1.5,2)--cycle;



               \draw (15,1) -- (15,1.2) -- (15.4,1.2) -- (15.4,1)--cycle;
               \draw (15.2,1) -- (15,0);
               \draw (15.2,1) -- (15.4,0);
               \draw (14.8,1.3)--(15,1.3)--(15,0.9)--(14.8,0.9)--cycle;
               \draw[dashed] (14.8,1.2) -- (5,10);
               \draw[dashed] (14.8,1) -- (5,0);

               \draw (2.2,11.85) circle [radius=0.3];
               \draw (2,11.7) -- (2,11.0) -- (1.8,10) --(2,11.0) -- (2.2,10);
               \draw (2,11.6) -- (2.2,11)--(2.3,11.6);
               \draw (2,11.53) -- (2.15,11.1)--(2.32,11.6);

               \node at (2,13.1) {\ldots};
               \draw (2,12.3)--(2,12.9);
               \node at (5.9,13.2) {Because I can?};
               \draw(4.85,12.3) -- (5.9,13);

               \draw[dashed] (5.6,7.58) -- (5.6,6.5);
               \centerarc[dashed](5.6,7)(-180:0:0.5)

               \draw[thick] (5.6,6.58) -- (5.6,5.5);
               \centerarc[thick](5.6,6)(-180:0:0.5)
               \draw[very thick,red] (5.6,6.5)--(1.8,6.5);
               \draw[very thick,red] (5.6,5.5)--(1.8,5.5);

               \draw[thick,<->,blue] (2,6.5)--(2,5.5);
               \node at (1.8,6) {$d$};
               \node at (6.6,6) {$t_1$};
               \node at (6.6,7) {$t_0$};
               \node at (6,5) {$v=\frac{d}{\Delta t}$};

    \end{tikzpicture}}\\

    Ideally, of course, one would be dropping a sphere, as even if it tumbles, the frontal surface area is constant, but an umbrella is a reasonable approximation commonly found at home. \\

    Thus, we have been able to calculate the drag coefficient of an umbrella.
    Not sure why someone would want this, but there you go.
\end{document}